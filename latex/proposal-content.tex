\section{Introduction}
Since the introduction of word2vec~\cite{mikolov2013efficient, mikolov2013distributed}, blah blah

\subsection{Problem Statement}
The project will investigate how to...

\subsection{Prior Work}
Mention (at least)~\cite{levy2014linguistic, mikolov2013exploiting, wolf2014joint}

Describe both the broader spectrum of word2vec/NLP \textbf{and} machine translation. Position our research (i.e. we skip grammar), and be sure to limit our project to per-word translation (not full grammatically correct sentences).

\lipsum[66]

\section{Research goal}
\lipsum[66]

\subsection{Research Purpose}
Abstract description of goals

\lipsum[66]

\subsection{Goals}
Pretty much what our deliverables are...

\subsection{Tasks}
What do we have to do to achieve this?
\begin{itemize}
  \item Build (at least) two working setups for automated translation
    \begin{itemize}
      \item For the single-model setup
      \begin{itemize}
        \item Merge datasets, train single model
      \end{itemize}
      \item For the multi-model setup
      \begin{itemize}
        \item Train individual models
        \item Train transformation matrix between models
      \end{itemize}
    \end{itemize}
  \item Make a verification mechanism
  \item Verify which setup works better
\end{itemize}

\section{Planning}
A global overview of the milestones we defined in our research is as follows;
\begin{description}
  \item \emph{Half September - October: Literary Study)}\\
    During this phase, a list of relevant papers (e.g. \cite{levy2014linguistic, mikolov2013exploiting, wolf2014joint}) is collected, short-listed to a readable size and read. Each paper will get a brief (informal) summarization to capture the essence of the paper insofar as it is relevant to this research. These summaries help process the information and provide quick access during the practical research.\\

    Not all papers relevant to the research will be read during this phase, so next phases will include a fair share of reading.
  \item \emph{October: Preparing experiments}\\
    Description
  \item \emph{November: Running experiments}\\
    Description
  \item \emph{Before 12 December: Incorporating results in paper}\\
    During this phase, we will incorporate the results obtained in the experiments into our final paper.
\end{description}

\section{Allocation of Responsibilities}
Marc pledges to provide sufficient beer. Bram will take care of coffee and possibly some Club-Maté.

\section{Resources}
We have identified three main resources needed for this research: a large English corpus, a large Dutch corpus and computional resources to analyze the data.

The two corpora should ideally be several tens of gigabytes or larger. Mikolov et al.~\ref{mikolov2013distributed} specifically state that a large amount of training data is crucial for word2vec to build a correct model. They speak of corpora 30 billion words, with a significant decrease in correctness when lowering the corpus size to only 6 billion words.

For this research, we will look at the following corpora:
\begin{itemize}
  \item \textit{Reddit corpus}
    \footnote{\url{https://archive.org/details/2015_reddit_comments_corpus}}
    (1TB uncompressed JSON, 50 billion words) Corpus containing all posts on Reddit. It is mostly English, but also contains a few other languages.
  \item \textit{Wikipedia EN crawl} 
    \footnote{\url{https://dumps.wikimedia.org/enwiki/20150901/}}
    (60GB uncompressed XML, 3 billion words) Corpus of all text on the English Wikipedia (ignoring revisions).
  \item \textit{Wikipedia NL Crawl} 
    \footnote{\url{https://dumps.wikimedia.org/nlwiki/20150901/}}
    (5GB uncompressed XML, 250 million words) Corpus of all text on the Dutch Wikipedia (ignoring revisions).
  \item \textit{SoNaR-500} 
    \footnote{\url{http://tst-centrale.org/producten/corpora/sonar-corpus/6-85}}
    (60GB uncompressed text, >500 million words) Curated corpus gathered by several Dutch universities, consisting of newspapers, subtitles, etc.
\end{itemize}

These datasets are all open for academic usage, in various degrees of opennes.